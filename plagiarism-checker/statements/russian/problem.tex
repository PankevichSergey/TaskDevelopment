\begin{problem}{Проверка на списывание}{стандартный ввод}{стандартный вывод}{1 секунда}{256 мегабайт}

Вы создали очередную платформу для проведения онлайн соревнований по программированию. После первого контеста вы изучили решения участников и заметили, что некоторые списывают друг у друга. После этого вам захотелось написать программу, чтобы быстрее обнаруживать списывания и блокировать нечестных участников. 

Пусть есть две строки, описывающие решения. Рассмотрим не более одного раза каждый символ, хотя бы где-то входящий в первую строку. После этого для рассматриваемого символа $x$ определим другой символ $p(x) \neq x$ и заменим некоторые вхождения $x$ в первое решение на $p(x)$. Если в ходе такого процесса из первого решения возможно получить второе, то скажем, что второе решение списано с первого. 

Иными словами, участник копирует чужое решение и, чтобы списывание не было таким очевидным, один раз рассматривает некоторые различные символы $x$, которые хотя бы раз встречаются в первом решении. Для каждого такого символа $x$ он выбирает, на какой другой символ $p(x)$ он будет заменять его вхождения, после чего проходит по строке и заменяет в ней $x$ на $p(x)$, но на некоторых позициях забывает это сделать (или там замена невозможна).

Значения $p(x)$ участником выбираются независимо для разных $x$, поэтому они могут и совпасть.

Напишите программу, которая позволит обнаружить списывания такого рода.

\InputFile
Кроме платформы вы создали язык программирования S++ и, чтобы его популяризировать, вы решили разрешить сдавать задачи в своей системе только на нем. Одной из особенностью языка является то, что любая программа записывается в одну строку и может состоять только из строчных и заглавных букв английского алфавита (a-z, A-Z), цифр (0-9), скобок (`(', `)', `\{', `\}' `[', `]'), знаков сравнения (`<', `>', `=') и символов `+', `-', `*', `/', `;'. 

В первой строке вам дано число  $n$ $(1 \leqslant  n \leqslant 200\,000)$, равное длине каждой из программ.

Во второй строке вам дана программа первого участника на языке S++.

В третьей строке вам дана программа второго участника на языке S++.


\OutputFile
Если вторая программа списана с первой, выведите <<YES>> (без кавычек), на следующей строке выведите $m$ -- количество различных символов в первой программе, которые хотя бы раз заменялись. Обозначим эти символы за $c_1,\ c_2,\ \ldots, \  c_m$. После этого выведите $m$ строк. На $i$-й строке  необходимо вывести символы $c_i$ и $p(c_i)$ через пробел. Если вторая программа не списана с первой, то выведите <<NO>> (без кавычек).


\Examples

\begin{example}
\exmpfile{example.01}{example.01.a}%
\exmpfile{example.02}{example.02.a}%
\exmpfile{example.03}{example.03.a}%
\exmpfile{example.04}{example.04.a}%
\exmpfile{example.05}{example.05.a}%
\exmpfile{example.06}{example.06.a}%
\end{example}

\Note
В первом примере списывающий заменил все вхождения `i' на `j' и вхождения `n' на `m'. 

Во втором примере участник заменял `a' на `e', `b' на `f', `с' на g' и `-' на `+'. 

В третьем примере невозможно осуществить процесс замен так, чтобы из первой строки получилась вторая. 

В четвертом примере участник списал, заменив некоторые вхождения `a' на `b'. 

В пятом примере участник списал, заменив некоторые вхождения `a' на `b' и все вхождения `c' на `a'.

\Scoring
В данной задаче $50$ тестов, помимо тестов из условия, каждый из них оценивается в $2$ балла. Результаты работы ваших решений на всех тестах будут доступны сразу во время соревнования. 

Решения, корректно работающие, когда обе программы состоят только из символов `a' и `b', наберут не менее 20 баллов.

Решения, корректно работающие, когда обе программы состоят только из символов `a', `b', `c', наберут не менее 40 баллов.

Решения, корректно работающие при $n \leqslant 1\,000$ наберут не менее 60 баллов.

Решения, корректно работающие только для случая, когда обнаружить списывание не удалось, оцениваются в 0 баллов.




\end{problem}

