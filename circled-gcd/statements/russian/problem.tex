\begin{problem}{НОД по кругу}{стандартный ввод}{стандартный вывод}{2 секунды}{256 мегабайт}

Массив из $n$ натуральных чисел $a_1,\ a_2,\ \ldots a_n$ записан по кругу. Следующим для числа $a_i$ назовем число $a_{i + 1}$, если $1 \leq i \leq n - 1$. Для числа $a_n$ следующим является число $a_1$ .

По массиву $a$ можно построить массив $b$ длины $n$, где $b_i$ равно наибольшему общему делителю $a_i$ и следующего за ним по кругу числа. Более формально, $b_i =$ НОД$(a_i, a_{(i \mod n) + 1})$. После этого массив $a$ заменяется на массив $b$.

Какое минимальное число таких операций необходимо совершить, чтобы все числа в массиве стали равны?

\InputFile
В первой строке дано натуральное число $n$ $(2 \leqslant  n \leqslant 200\,000)$. В следующей строке через пробел заданы $n$ натуральных чисел: $a_1,\ a_2,\ldots, a_n$ $(1 \leqslant a_i \leq 1\,000\,000)$

\OutputFile
В единственной строке выведите целое число~--- сколько операций нужно совершить, чтобы все числа массива стали равными.

\Examples

\begin{example}
\exmpfile{example.01}{example.01.a}%
\exmpfile{example.02}{example.02.a}%
\exmpfile{example.03}{example.03.a}%
\end{example}

\Note
В первом примере после одной операции массив превратится в $(2,\ 1,\ 1)$, после следующей в $(1,\ 1,\ 1)$.

Во втором примере все числа изначально равны, поэтому не нужно совершать никаких операций.

Можно показать, что в третьем примере необходимо совершить ровно 4 операции.

\end{problem}

