Массив из $n$ натуральных чисел $a_1,\ a_2,\ \ldots a_n$ записан по кругу. Следующим для числа $a_i$ назовем число $a_{i + 1}$, если $1 \leq i \leq n - 1$. Для числа $a_n$ следующим является число $a_1$ .

По массиву $a$ можно построить массив $b$ длины $n$, где $b_i$ равно наибольшему общему делителю $a_i$ и следующего за ним по кругу числа. Более формально, $b_i =$ НОД$(a_i, a_{(i \mod n) + 1})$. После этого массив $a$ заменяется на массив $b$.

Какое минимальное число таких операций необходимо совершить, чтобы все числа в массиве стали равны?
